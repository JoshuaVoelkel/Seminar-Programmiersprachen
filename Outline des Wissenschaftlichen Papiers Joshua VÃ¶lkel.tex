% Options for packages loaded elsewhere
\PassOptionsToPackage{unicode}{hyperref}
\PassOptionsToPackage{hyphens}{url}
%
\documentclass[
]{article}
\usepackage{amsmath,amssymb}
\usepackage{lmodern}
\usepackage{iftex}
\ifPDFTeX
  \usepackage[T1]{fontenc}
  \usepackage[utf8]{inputenc}
  \usepackage{textcomp} % provide euro and other symbols
\else % if luatex or xetex
  \usepackage{unicode-math}
  \defaultfontfeatures{Scale=MatchLowercase}
  \defaultfontfeatures[\rmfamily]{Ligatures=TeX,Scale=1}
\fi
% Use upquote if available, for straight quotes in verbatim environments
\IfFileExists{upquote.sty}{\usepackage{upquote}}{}
\IfFileExists{microtype.sty}{% use microtype if available
  \usepackage[]{microtype}
  \UseMicrotypeSet[protrusion]{basicmath} % disable protrusion for tt fonts
}{}
\makeatletter
\@ifundefined{KOMAClassName}{% if non-KOMA class
  \IfFileExists{parskip.sty}{%
    \usepackage{parskip}
  }{% else
    \setlength{\parindent}{0pt}
    \setlength{\parskip}{6pt plus 2pt minus 1pt}}
}{% if KOMA class
  \KOMAoptions{parskip=half}}
\makeatother
\usepackage{xcolor}
\setlength{\emergencystretch}{3em} % prevent overfull lines
\providecommand{\tightlist}{%
  \setlength{\itemsep}{0pt}\setlength{\parskip}{0pt}}
\setcounter{secnumdepth}{-\maxdimen} % remove section numbering
\ifLuaTeX
  \usepackage{selnolig}  % disable illegal ligatures
\fi
\IfFileExists{bookmark.sty}{\usepackage{bookmark}}{\usepackage{hyperref}}
\IfFileExists{xurl.sty}{\usepackage{xurl}}{} % add URL line breaks if available
\urlstyle{same} % disable monospaced font for URLs
\hypersetup{
  hidelinks,
  pdfcreator={LaTeX via pandoc}}

\author{}
\date{}

\begin{document}

\textbf{Outline}

\textbf{1. Einleitung:}

In der Einleitung wird erklärt, was die Programmieraufgabe war und wovon
die wissenschaftliche Arbeit handelt

\textbf{2. Die Programmiersprache Rust}

\textbf{2.1. Geschichte und Motivation}

Es wird die Geschichte von Rust erzählt, wie es zur Entwicklung kam und
was Rusts Zukunftspläne sind. Außerdem wird über die Motivation zur
Kreation von Rust geschrieben und es werden kurz ein paar aktuelle
Projekte, welche Rust benutzen, vorgestellt (1 Projekt mit 2-3 Sätzen,
der Rest mit je einem halben Satz).

\textbf{2.2 Besonderheiten der Programmiersprache Rust}

Es wird ausführlich auf die Besonderheiten Rusts eingegangen. Dabei
werden die besonderen Eigenschaften (Memory Safety ohne Gargabe
Collection -- Ownership System, Zero Cost Abstractions, Fearless
Concurrency und als eine interessante Mechanik unsafe Rust vorgestellt
und ausführlich erklärt).\\
Des Weiteren werden wird mit Cargo, (Unterpunkte: Paket-Management,
Dependency Management, weitere Funktionen), der Rust Dokumentation,
Rustdoc, Rustc (der Rust Compiler), Rustup, ein paar IDEs und der Rust
Community alles zur Infrastruktur um Rust herum erläutert.

\textbf{3. Das Programm}

\textbf{3.1 Die Libraries}

Es werden einige der wichtigen benutzten Libraries (Rust
Standardbibliothek, Regex, Walkdir, Rayon) ausführlich erläutert, einige
der häufig oder an wichtigen Stellen benutzten Funktionen der
entsprechenden Libraries beschrieben und teilweise Codestellen als
Beispiele mit angegeben. Außerdem wird erklärt, wieso zum
Parallelisieren Rayon anstatt der Standardbibliothek verwendet wurde.

\textbf{3.2 Schwierigkeiten mit der Programmieraufgabe/Optimierung des
Programms}

Die größte Schwierigkeit mit der Programmieraufgabe war die lange
Laufzeit des Programms und das Arbeiten mit Subtitles.txt. Es wird
beschrieben, wie diese Probleme mit Hilfe von Batches und Chunks gelöst
wurden, wie das Programm parallelisiert wurde (falls dies nicht schon in
Abschnitt 3.1 geschieht) und wie dies das Programm optimiert hat (es
wurde schneller und hat weniger Ram verbraucht). Es wurde auch das Tool
Memcheck zum Überprüfen von Memory Leaks benutzt, jedoch wurden bei der
Überprüfung keine gefunden, was an Rusts Eigenschaften liegen sollte.

Als Ergänzung wird je nach Wortanzahl noch erwähnt, wie der Rust
Compiler und Fehlerindex bei der Lösung von Fehlern stark geholfen und
das Arbeiten mit Rust sehr vereinfacht hat~

\textbf{4. Fazit\\
}Das Fazit wird von der Erfahrung und den Vor- und Nachteilen von Rust
in Bezug auf die Implementierung des Programms und beim generellen
Programmieren handeln

// 2 Kommentare:

1. Zu rein dem Code und der Umsetzung der Aufgabe an sich kann man
meiner Meinung nach nicht besonders viel schreiben, was der
wissenschaftlichen Arbeit Tiefe verleihen würde (in dem Sinne Problem X
(z.~B. die Kontextzeilen) habe ich mit Code Y gelöst). Eine von Rusts
Stärken ist schließlich die große Anzahl an hilfreichen Libraries, über
welche deswegen auch ein Großteil von Punkt 3 gehen wird und da dadurch
schon die wichtigen Punkte des Codes (z.B. die Eingabe und Ausgabe in
die Konsole oder das Durchlaufen der Ordner) abgedeckt werden, wüsste
ich nicht, was man sonst noch großartig schreiben sollte

2. Ich habe das parallele Problem doch noch selbst lösen können, wir
müssen also am 6. Dezember nicht mehr darüber reden.

\end{document}
