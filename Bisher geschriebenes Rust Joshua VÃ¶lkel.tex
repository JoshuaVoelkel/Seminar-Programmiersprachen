% Options for packages loaded elsewhere
\PassOptionsToPackage{unicode}{hyperref}
\PassOptionsToPackage{hyphens}{url}
%
\documentclass[
]{article}
\usepackage{amsmath,amssymb}
\usepackage{lmodern}
\usepackage{iftex}
\ifPDFTeX
  \usepackage[T1]{fontenc}
  \usepackage[utf8]{inputenc}
  \usepackage{textcomp} % provide euro and other symbols
\else % if luatex or xetex
  \usepackage{unicode-math}
  \defaultfontfeatures{Scale=MatchLowercase}
  \defaultfontfeatures[\rmfamily]{Ligatures=TeX,Scale=1}
\fi
% Use upquote if available, for straight quotes in verbatim environments
\IfFileExists{upquote.sty}{\usepackage{upquote}}{}
\IfFileExists{microtype.sty}{% use microtype if available
  \usepackage[]{microtype}
  \UseMicrotypeSet[protrusion]{basicmath} % disable protrusion for tt fonts
}{}
\makeatletter
\@ifundefined{KOMAClassName}{% if non-KOMA class
  \IfFileExists{parskip.sty}{%
    \usepackage{parskip}
  }{% else
    \setlength{\parindent}{0pt}
    \setlength{\parskip}{6pt plus 2pt minus 1pt}}
}{% if KOMA class
  \KOMAoptions{parskip=half}}
\makeatother
\usepackage{xcolor}
\setlength{\emergencystretch}{3em} % prevent overfull lines
\providecommand{\tightlist}{%
  \setlength{\itemsep}{0pt}\setlength{\parskip}{0pt}}
\setcounter{secnumdepth}{-\maxdimen} % remove section numbering
\ifLuaTeX
  \usepackage{selnolig}  % disable illegal ligatures
\fi
\IfFileExists{bookmark.sty}{\usepackage{bookmark}}{\usepackage{hyperref}}
\IfFileExists{xurl.sty}{\usepackage{xurl}}{} % add URL line breaks if available
\urlstyle{same} % disable monospaced font for URLs
\hypersetup{
  hidelinks,
  pdfcreator={LaTeX via pandoc}}

\author{}
\date{}

\begin{document}

//Kommentar: Dies ist nur, was ich bisher geschrieben habe, falls du
Interesse hast, dir das durchzulesen. Ich habe es zwar online auf
Rechtschreibung und Grammatik überprüft, aber nicht noch einmal auf
falsche/doppelte Worte, die Fomulierung oder den Sinn überprüft .
Quellenangaben habe ich auch nicht nicht hinzufügt (ich hab jedoch alle
Quellen an einem sicheren Ort abgespeichert, damit ich das später noch
machen kann)

\textbf{Die Programmiersprache Rust}

\textbf{Die Geschichte} \textbf{von Rust}

Die Entwicklung Rusts startete 2006 als persönliches Projekt des Mozilla
Mitarbeiters Greaygon Hoare. Daraufhin begann Mozilla, die Entwicklung
von Rust im Rahmen eines experimentellen Browsers namens Servo
finanziell zu unterstützten, was schließlich dazu führte, dass die
Programmiersprache 2010 offiziell in einer Präsentation ankündigt wurde.
Nach weiterer Entwicklung kompilierte sich der Rust Compiler 2011 das
erste Mal und am 15. Januar 2015 wurde Rust 1.0 offiziell angekündigt.

\textbf{Die aktuelle Situation und Entwicklung von Rust}

Die aktuelle Situation Rusts ist durch einen großen Zuwachs an
Entwicklern geprägt. Laut der State of the Developer Nation 22nd Edition
hat sich alleine innerhalb der letzten 2 Jahre die Anzahl an Mitgliedern
in der Rust Community von 600.000 auf 2,2 Millionen erhöht und somit
fast vervierfacht.

Außerdem wurde am 8. Februar 2021 die Rust Foundation gegründet, eine
Non-Profit-Organisation, welche sich dafür einsetzt, die an Rust
beteiligten Entwickler zu unterstützen. Dabei kommen die Direktoren der
Rust Foundation aus den 5 Gründerfirmen, namentlich Google, Mozilla,
AWS, Huawei und Microsoft, welche jährlich 300.000 € spenden.

Die Zukunftspläne von Rust bestehen aus drei großen Zielen, nämlich den
Einstieg zu vereinfachen und die Lernkurve zu senken, es Nutzern besser
zu ermöglichen, sich untereinander zu helfen, und das gesamte Projekt
der Weiterentwicklung Rusts übersichtlicher zu machen. Für die Senkung
der Lernkurve plant man unter anderem, den Compiler zu verbessern und
die Sprache zu erweitern, damit man durch Code besser vermitteln kann,
was man möchte. Das Ziel, das Nutzer sich untereinander besser helfen
können, soll durch eine Verbesserung des Librarysystems und den Support
von Libraryauthoren geschehen und die Weiterentwicklung Rusts soll durch
klare Problemstellungen mit einer Angabe der nächsten erforderlichen
Schritte und eine eindeutige Aufgabenzuweisung erfolgen.

Die aktuelle Version von Rust (Stand 16. November 2023) ist Version
1.74.0.

\textbf{Motivation zur Entwicklung von Rust}

Die Motivation Rust zu entwickeln bekam Greaygon Hoare als zu Hause in
den 21. Stock laufen musste, da der Aufzug nicht funktionierte und er
sich fragte, wieso man nicht mal einen fehlerfrei laufenden Aufzug
programmieren könnte. Aufgrund dieses Vorfalls begann Hoare weiter über
den damaligen Zustand von C++ nachzudenken, woraufhin ihm C++s Schwächen
wie zum Beispiel unsichere Memorybehandlung oder keine Kontrolle über
Parallelität in den Sinn kamen. Hinzu kam, dass zu dem damaligen
Zeitpunkt die neu erschienen Sprachen alle die Nische, in der sich C++
befand, gemieden haben, weswegen man trotzdem auf C++ zurückgreifen
musste. Dies führte schließlich zu der Idee, eine Programmiersprache
entwickeln zu wollen, die Sicherheit, Funktionalität und Effizienz an
erste Stelle stellt.

\textbf{Wozu wurde Rust entwickelt und in welchen Bereichen benutzt man
Rust?}

Rust wurde entwickelt, um eine sicherere Alternative zu C++ zu sein,
welche jedoch eine mindestens genauso schnelle und gut Performance
bietet. Aufgrund dessen kann Rust in einer Vielzahl an
Programmierbereichen angewendet werden, jedoch hat man sich 2018 auf
vier Bereiche fokussiert, für die man Rust besonders verbessert:
Kommandobefehle, Web-Assembly, Netzwerke und den Bereich Embedded
Systems.

\textbf{Von wem wird Rust heutzutage benutzt?}

Rust wird aktuell von über 100 verschiedenen Firmen, meistens welche mit
dem Fokus auf Softwareentwicklung, verwendet. Ein Beispiel eines großen
Projektes, welches Rust benutzt, ist Mozilla Firefox. Nachdem Mozilla
bei der Erstellung von Rust mitgeholfen hat, benutzen sie die
Programmiersprache auch in ihrem Server. Dort wird Rust für die Stylo,
CSS Engine des Browsers, benutzt. Dadurch sind nun etwas mehr als 10\%
des gesamten Firefox-Codes in Rust geschrieben. Weitere Beispiele großer
Projekte mit Rust Nutzung sind das Android Open-Source-Projekt, in
welches Rust innerhalb der letzten paar Jahre integriert wurde, das
Designtool Figma, welches aufgrund der nicht mehr ausreichenden
Serverleistung mit TypeScript in Rust umgeschrieben wurde, die
Kommunikationsapp Discord, welche zur Performanceverbesserung von Go zu
Rust wechselte und das Betriebssystem Linux, welches seit der Version
6.1 jetzt auch Rust Support besitzt.

\textbf{Die Besonderheiten der Programmiersprache Rust}

\textbf{Besondere Eigenschaften}

\textbf{Memory Safety ohne Garbage Collector -- Das Eigentümersystem}

Im Gegensatz zu vielen anderen Sprachen benötigt Rust weder Memory
Safety, wobei das Programm Memory explizit zuordnen und freistellen
muss, noch Garbage Collection, wobei ständig nach nicht mehr genutzter
Memory Ausschau gehalten wird. Rust benutzt dafür das Eigentümersystem.
Dies bedeutet, dass eine Variable jederzeit genau einen Eigentümer
besitzt. Solange dieser Eigentümer gültig ist, ist auch die Variable
gültig, danach verfällt sie automatisch wieder, wodurch weder Memory
Safety noch Garbage Collection nötig sind.

Bei dem oben genannten Prinzip tritt natürlich die Frage auf, was
passiert, wenn man eine neue Variable mit dem Wert einer bereits
erstellen Variable initialisiert, zum Beispiel erstellt man erst x1 und
initialisiert dann x2 = x1. Dabei können je nach Datentyp zwei
unterschiedliche Ereignisse in Kraft treten: Sollte die Variable einen
Wert besitzen, welcher nicht einmalig ist (zum Beispiel eine Zahl), dann
würde die neu erstellte Variable lediglich den Wert der alten kopieren
und es würden zwei Variablen gleichen Wertes existieren. Sollte jedoch
ein einmaliger Datentyp im Spiel sein, wie ein erstellter String, dann
würden, sollte die neue Variable nur die Werte der Alten kopieren, beide
Variablen auf den gleichen String zugreifen, was nachdem beide Variablen
nicht mehr gültig sind, zu Problemen führen würde. Dieses Problem wird
gelöst, in dem die neue Variable den Wert der früher erstellen erhält
und diese danach ungültig wird. So existiert weiterhin nur eine Variable
mit dem gleichen, einmaligen, Wert.

Sollte man jedoch einzigartige Datentypen, wie einen String, kopieren
wollen, gibt es dafür die clone()- Methode, welche entsprechende
Datentypen kopiert.

\textbf{Zero-Cost Abstractions}

Zero-Cost Abstractions bedeutet, dass man ein hohes Level an
Abstraktionen, zum Beispiel durch Funktionen, in seinem Code
unterbringen kann und diese keine Runtimekosten verursachen. Dies führt
dazu, dass Entwickler komplexe Ideen umsetzten und ein hohes Codelevel
verwenden könne, ohne dass dabei zusätzliche Kosten entstehen. Dies
funktioniert aufgrund der Kombination von Rusts Memoryverhalten und der
Arbeitsweise von Rusts Compiler, welcher allen Code in Maschinensprache
konvertiert.

\textbf{Fearless Concurrency}

Eine von Rusts Eigenschaften ist es, dass Programmierer problemlos
parallel programmieren können, ohne auf die bei anderen
Programmiersprachen vorhandenen Probleme wie Race Conditions, wobei zwei
Threads zur gleichen Zeit auf eine Variable zugreifen, Deadlocks, wobei
mehrere Threads mehrere, gleiche Variablen brauchen und jeder auf einen
Teil der benötigten Variablen zugreift, wodurch beide Threads nun auf
die Variablen, die der jeweils andere Thread besitzt, zugreifen müssen
und, da keiner der Threads nachgibt, das Programm an der Stelle
festsitzt.

Rusts Standardbibliothek alleine unterstützt 4 verschiedene
Möglichkeiten, Parallelismus zu nutzen.

Variante eins ist die alleinige Nutzung von Threads. Dabei werden mit
Hilfe von thread::spawn threads erzeugt, welche nach der Bewältigung
ihrer Aufgabe schlafen gelegt werden. Am Ende der Methode oder Schleife
kann durch die join() Methode dafür gesorgt werden, dass die Threads
warten, bis alle Threads ihre Aufgabe beendet haben.

Die zweite Möglichkeit ist das Senden von Nachrichten zwischen
verschiedenen Threads. Die Idee dieser Möglichkeit stammt dabei von der
Programmiersprache Go, welche das Konzept berühmt machte. In Rust wird
für das Senden von Nachrichten zuerst ein Channel erstellt, durch welche
die Nachrichten später geschickt werden. Jeder Channel besteht dabei aus
dem Sender, welcher die Nachricht sendet und den Empfänger, welcher die
Nachrichten empfängt. Durch das Aufrufen von entsprechenden Methoden
werden die Nachrichten dann durch den Channel vom Sender geschickt und
vom Empfänger empfangen. Dabei können die geschickten Nachrichten nicht
nur Textnachrichten sein, sondern zum Beispiel auch Values. Bei
verschickten Values gibt der Sender sein Eigentum der Variable an den
Empfänger weiter und kann danach nicht mehr auf diese zugreifen. Der
Channel wird geschlossen, sobald entweder Sender oder Empfänger nicht
mehr gültig sind.

Variante Drei um mit Parallelität zu arbeiten ist die Shared-State
Parallelität. Dabei wird ein Datenpool erstellt, auf den alle Threads
über Mutexe zugreifen können. Mutex ist die Abkürzung für Mutual
Exclusion, was bedeutet, dass zum Zeitpunkt X nur ein Thread A auf eine
Ressource zugreifen kann. Möchte ein anderer Thread B auch auf diese
Ressource zugreifen, dann muss er warten. Dies ist möglich, indem Thread
A vorher anfragt, ob er das sogenannte Lock für die Ressource haben
kann, wodurch diese nur für ihn zugänglich wird. Nachdem Thread A seine
Aufgabe, für die die Ressource benötigt wurde, erfüllt hat, gibt er das
Lock wieder ab und Thread B kann nun das Lock bekommen und seine Aufgabe
erfüllen. Bei der Shared-State Parallelität wird genau dieses Prinzip
der Mutual Exclusion auf den Pool von Ressourcen angewandt.

Möglichkeit 4 des Umgangs mit Parallelität sind die Eigenschaften Send
und Sync. Diese befinden sich, im Vergleich zu den Varianten 1-3, nicht
in der Standardbibliothek, sondern sind direkt in die Sprache
integriert. Send und Sync sind dabei Eigenschaften, welche alle Values
erhalten. Sent steht dafür, dass ein Value zwischen zwei Threads
verschoben werden darf, wobei wie beim Senden in Möglichkeit Zwei das
Eigentum übergeben wird. Sync bedeutet, dass ein Value von mehreren
Threads bearbeitet werden kann, aber immer nur von einem gleichzeitig,
womit es praktisch wie ein Mutex funktioniert. Die Send und Sync
Eigenschaften werden automatisch an die entsprechenden Values vergeben
und es wird empfohlen, diese nicht zu ändern.

\textbf{Unsafe Rust}

Da bei Rust die Sicherheit des Programms im Vordergrund steht, gibt der
Compiler, falls er sich unsicher ist, ob der Code funktioniert, eine
Warnung aus. Wenn man aber weiß, was man tut, oder es für das Programm
erforderlich ist, kann man durch die Verwendung des Schlüsselwortes
„unsafe`` in Rusts unsicheren Modus wechseln. Im unsicheren Modus kann
man 5 Aktionen ausführen, die sonst nicht möglich sind: Man kann rohe
Pointer (Pointer, die nicht gültig sein müssen, null sein können und die
normalen Pointeregeln ignorieren) dereferenzieren, eine unsichere
Funktion oder Methode aufrufen, auf eine veränderbare, statische,
Variable zugreifen oder sie verändern, eine unsichere Eigenschaft (Send
und Sync) implementieren und auf das Feld einer Union zugreifen. Sollte
man das unsichere Rust benutzen, checkt der Compiler zwar trotzdem den
gesamten Code, der unsichere Teil des Codes wird jedoch nicht auf
Memoryfehler überprüft.

\textbf{Die Infrastruktur von Rust}

\textbf{Cargo}

\textbf{Paket Management mit Cargo}

Das Paketmanagement wird in Rust mit einem Paket-Manager namens Cargo
durchgeführt.

Libraries fallen in Rust unter den Oberbegriff Crates, wobei bei diesen
zwischen einer Library Crate und einer Binary Crate unterschieden wird.
Library Crates sind Rusts Äquivalente zu Libraries in anderen Sprachen,
während Binary Crates eigene Programme mit einer main() Funktion, welche
man kompilieren kann, sind.~

Cargo ist dafür zuständig, dass man in Rust problemlos Crates verwenden
kann. Um dies zu gewährleisten, übernimmt Cargo für jedes Paket die
folgenden 4 Aufgaben: Es verwaltet die Metadaten des Pakets, es holt die
Pakete und kümmert sich um die benötigten Abhängigkeiten, es ruft ein
anderes Tool auf, welches die Möglichkeiten besitzt, mit den geholten
Paketen umzugehen und es macht das Arbeiten mit den Paketen innerhalb
von Rust durch eine einheitliche Paketverwaltung einfacher. Außerdem
hilft Cargo auch bei dem Arbeiten an Paketen, da es die Option besitzt,
für selbst erstellte Pakete direkt ein Git-Repository zu errichten und
diese zu kompilieren.

\textbf{Dependency Management mit Cargo}

Abhängigkeiten werden mit zwei Dateien namens Cargo.toml und Cargo.lock
gemanagt. Cargo.toml ist dabei eine vom Nutzer geführte Datei, in welche
für alle genutzten Pakete ihre Abhängigkeiten und ihre aktuelle Version
geschrieben werden, während Cargo.lock von Cargo selbst geführt ist und
spezifische Informationen zu den Abhängigkeiten aller genutzten Pakete
erhält. Es gibt pro Programm, insofern es Pakete benutzt, sowohl einmal
Cargo.toml als auch Cargo.lock und beide werden in der Root des Projekts
gespeichert.

\textbf{Weitere Funktionen von Cargo}

Cargo verfügt über eine Vielzahl von Funktionen, von welchen im
Folgenden einige erläutert werden.

Eine wichtige Funktion Cargos ist die Testunterstützung für Benutzer.
Mit Hilfe von Cargo können Benutzer einfach Tests, wie Binary Tests,
Unit Tests, Integration Tests oder Benchmarks schreiben und diese mit
Hilfe der cargo test Operation testen.

Eine weitere Nutzungsmöglichkeit Cargos ist das Kompilieren von Paketen.
Mit der cargo build Funktion können Pakete direkt von Cargo mit all
ihren Abhängigkeiten kompiliert werden.

Außerdem ist es möglich, mit Hilfe von Cargo das aktuelle Paket auf
Fehler zu überprüfen (cargo check), die Dokumentation für ein Paket zu
erstellen (cargo doc), ein Programm oder ein Beispiel für ein Paket
laufen zu lassen (cargo run) und alle Abhängigkeiten für ein Paket zu
holen und sicherzustellen, dass diese in Cargo.lock vorhanden sind und
es keine Cargo.lock gibt, die entsprechende Datei zu erstellen (cargo
fetch).

Crates.io

Crates.io ist die Hauptseite zum Veröffentlichen, Finden und
Herunterladen von Rust Crates. Sie besitzt über 130.000 Crates mit
stündlichen Neuzugängen und nahezu minütlichen Updates von bereits
vorhandenen Crates. Wenn man Cargo zum Herunterladen von Crates nimmt,
werden diese auch automatisch von Crates.io heruntergeladen.

\textbf{Die Rust Dokumentation}

Rust besitzt eine, aus sechs Teilen bestehende, ausführliche
Dokumentation, welche sich auf der offiziellen Rust Homepage finden
lässt. Der erste Abschnitt befasst sich dabei mit Rusts enormer
Standard-Bibliothek, auf welche im Programmierteil dieser Arbeit noch
genauer eingegangen wird. Teil Zwei handelt von den verschiedenen Rust
Versionen. Dieser Teil der Dokumentation hält fest, welche Versionen es
von Rust gegeben hat, wann diese erschienen und welche Änderungen die
jeweiligen Versionen mit sich brachten. Außerdem wird erklärt, wie man
sein Rust-Projekt von einer alten Version in eine neue überträgt und was
man dabei beachten muss. Bereich Drei der Rust Dokumentation geht über
Cargo. In diesem wird beschrieben, wie man Cargo installiert und die
ersten Schritte mit Cargo. Des Weiteren gibt es noch einen Guide zu
sämtlichen Cargo Funktionen, von denen einige bereits in dem
entsprechenden Abschnitt dieser Arbeit erklärt wurden. Segment Vier der
Dokumentation handelt von dem Rustdoc Tool, welches dabei hilft
Dokumentationen für geschriebenen Rust Code zu erstellen. Auf dieses
Tool wird im folgenden Abschnitt noch genauer eingegangen. Die fünfte
Sektion der Dokumentation handelt von dem Rust Compiler, welcher in
einem der folgenden Abschnitte noch besonders behandelt wird. Der
sechste und letzte Abschnitt der Rust Dokumentation ist der Fehlerindex.
In diesem gibt es zu jedem von Rusts knapp 800 registrierten Fehlern
einen Eintrag, in welchem der Fehler beschrieben wird, oft ein Stück
fehlerhafter Code gezeigt wird und häufig auch ein Lösungsansatz
vorgeschlagen wird.

Abseits von den offiziellen Dokumentationen gibt es auf Rust Homepage
noch den Link zu einem Einführungsbuch zu Rust. Dieses wurde von Steve
Klabnik und Carol Nichols mit Hilfe der Rust Community geschrieben und
behandelt die Installation von und die ersten Schritte mit Rust,
Informationen zu Rusts Besonderheiten, zum Beispiel das
Eigentümersystem, eine generelle Sammlung an Informationen für
Einsteiger und Fortgeschrittene und ein kleines Einstiegsprojekt,
welches einige von Rusts Mechaniken und Besonderheiten zeigt.

\textbf{Das Rustdoc Tool}

Rustdoc ist ein Tool zum Dokumentieren von Rust Projekten, welches man
zum Bedienen Cargos benutzt. Man kann mit Hilfe von Rustdoc eine
Dokumentationsdatei erstellen, in welche automatisch die Dokumentation
der programmierten Funktionen geschrieben wird. Dies geschieht, indem
der Benutzer seine Kommentare vor oder innerhalb einer Funktion speziell
markiert, welche dann im Zusammenhang mit der Funktion in der
Dokumentationsdatei erscheinen.

\textbf{Rustc}

Rustc ist der Rust Compiler, welcher von Rust selbst bereitgestellt
wird. Auch wenn zum Kompilieren häufig Cargo verwendet wird, erfolgt der
Prozess des Kompilierens meistens durch den Rust Compiler.

Eine Besonderheit von Rustc ist, dass es beim Kompilieren auch direkt
eine Nummer von Lints, welche es besitzt, durchläuft. Lints sind
Werkzeuge, welche helfen, den geschrieben Code zu verbessern und dem
Benutzer Warnungen senden, wenn es im Code Stellen gibt, welche diesen
verschlechtern, die jedoch keine Fehler sind. Der Rust Compiler besitzt
5 Level von Lints: Allow, Warn, Force-Warn, Deny und Forbid. Allow ist
dabei das kleinste Level und muss manuell aktiviert werden, bevor es dem
Benutzer Nachrichten sendet. Es warnt vor Fehlern, wie zum Beispiel dem
Fehlen einer Dokumentation. Warn und Force-Warn sind Stufe 2 und 3 von
Lints und geben Warnungen für Codeteile, welche zum Beispiel
unerreichbare oder ungenutzte Variablen oder Funktionen erhalten. Der
Unterschied zwischen Warn und Force-Warn ist, dass das Level von Warn
Lints in den Einstellungen deaktiviert werden kann und von Force-Warn
Lints nicht. Deny und Forbid Lints werden, wenn ignoriert, einen Fehler
produzieren. Der Unterschied zwischen ihnen ist derselbe wie zwischen
Stufe 2 und 3. Der Rustc kann außerdem das Level von Lints der Stufe 1,
2 und 4 anpassen und Gruppen von Lints aktivieren oder deaktivieren.

Die Plattformunterstützung des Rust Compilers lässt sich in drei Level
einteilen. Dabei gibt es das „guaranteed to work`` Level (Level 1), für
welches es offizielle Releases und regelmäßig Tests von Rust gibt, das
„guaranteed to build`` Level (Level 2), für welches es offizielle
Releases, aber keine regelmäßigen Tests von Rust gibt, und Level 3,
welches Rust an sich unterstützt, aber wofür nichts Offizielles von Rust
erscheint. Die drei Level lassen sich zudem in die Kategorie mit und
ohne Host Tool (natürlicher Support für unterstützende Tools wie Cargo)
unterscheiden.

\textbf{Rustup}

Rustup ist ein Tool für das Installieren und Updaten von Rust. Da Rust
auf vielen verschiedenen Plattformen releast wird, sorgt Rustup dafür,
dass man bei Updates die richtige und neueste Version herunterlädt.

\textbf{IDEs}

Rust besitzt eine Vielzahl von IDEs, darunter auch in berühmten
Codingprogrammen wie Eclipse, IntelliJ IDEA und Visual Studio Code,
welche allesamt Rust Support besitzen und für Rust gut funktionierende
Compiler haben oder sogar zusätzliche Tools, welche man sich
herunterladen kann, zum Beispiel den Rust-Analyzer in Visual Studio
Code.

Besonders hervorstechen tut unter den IDEs aber eine, und das ist
RustRover von JetBrains, eine für Rust konzipierte. Sie befindet sich
derzeit noch in Entwicklung, soll jedoch nach der Fertigstellung Hilfe
für das Schreiben, Verstehen und Analysieren von Code und Syntax,
Unterstützung für das Testen, Ausführen und Debuggen von Programmen,
Performance- und Datenanalyse, Unterstützung für das Management von
Projekten und Support für die Webentwicklung mit Rust anbieten.

\textbf{Die Rust Community}

Ein weiterer wichtiger Bestandteil ist die Community von Rust. Neben der
Hilfe, welche sie bei der Weiterentwicklung der Sprache leistet und der
Hilfe der durch die unzähligen von der Community entwickelten Crates
spielen auch die große Menge an Rust Foren eine wichtige Rolle.

Neben berühmten Kommunikationsplattformen wie Reddit und Stack Overflow
gibt es auf der Rust Homepage alleine eine Unmenge an Links zu Internet
Relay Chats, in denen man sich mit anderen Programmieren austauschen
kann. Diese Gespräche gehen von nationalen Chats über Diskussionen für
spezielle Themen wie Browser oder Tools, welche in Rust geschrieben
wurden, bis hin zu verschiedenen Chats für Teams, welche an Rust
mitarbeiten.

\textbf{Rust Playground}

Der Rust Playground ist eine Entwicklungsumgebung, welche sich besonders
dazu eignet, schnell Code auszuprobieren, wenn man gerade keinen
Compiler verfügbar hat, oder ihn mit anderen zu teilen. Rust Playground
verfügt dabei über die Fähigkeiten im Debug oder Release-Modus zu
kompilieren, es verfügt über die 100 beliebtesten Crates von Crates.io
und es besitzt eine Funktion, mit der man den geschriebenen Code auf
GitHub hochladen und speichern kann.

\textbf{Das Programm}

\textbf{Wichtige benutzte Libraries}

\textbf{Die Rust Standardbibliothek}

Die Rust Standardbibliothek ist eine Bibilothek aus der offiziellen Rus
Dokumentation. Sie funktionmiert dabei als eine Art Grundlage der
gesamten Programmierpsrache Rust und bietet allen Programmierenen,
unabhängig des Projektes, eine große Anzahl an verschiedenen
Unterstützungsmöglichkeiten, um das Coden in Rust so einfach, sicher und
effizient wie möglich zu gestalten. Dafür besitzt sie eine Vielzahl an
Kerntypen, wie zum Beispiel vec\textless T\textgreater, der Kerntyp
Vector, womit man einen Vector erstellen kann, wofür man sonst eine
extra Bibilothek bräuchte. Des weiteren vefügt die Standardbibliothek
von Rust noch über Operationen auf primitiven Typen, Standard Macros,
und Modulen. Dabei sind Module besonders vielseitig und gehen von
parallelem Programmieren durch die Erstellung von Threads mit
std::thread über das Abrbeiten mit In-/Output durch das std::io Modul
bis hin zur Fehlerbehandlung mit std::error.

Für jede der eben genannten gennanten Unterstützungsmöglichkeitenbietet
die Dokumentation der Rust Standardbibliothek eine große Menge an
weiteren Informationen auf den entsprechenden Seiten. Auf der Seite des
io Moduls, welches für das geschrieben Programm von enormer Wichtigkeit
war, findet man zum Beispiel die Liste aller modulspezifischen
Strukturen, Eigenschaften und Funktionen. Des weiteren werden die
Kernfunktionend es Moduls anhand eines kurzen Codeabschnitts
demonstiert, welche man direkt durch einen Link zu dem im vorherigen
Thema bereits beschriebenen Rust Playground ausprobieren.

\end{document}
